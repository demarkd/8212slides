% If any of you wants to put in a decent header, be my guest.
% This is just the one that I stole from my undergrad :P 
% The top section is all somewhat essential, the bottom is just
%   a random dump of commands I've used once.   -eric

% --------------------------------------------------------------------

\documentclass[Letter,12pt]{article}
\usepackage{amsmath,amssymb,amsthm}
\usepackage{graphicx,framed,enumerate}
\usepackage[margin=1in]{geometry}
\usepackage{fancyhdr}
\usepackage{microtype}
\usepackage{centernot}
\usepackage{dirtytalk}
\usepackage{enumitem}
\frenchspacing

\newcommand{\problem}[2]{
	\vspace{0.3in} 
	\begin{leftbar} 
		\noindent \textbf{{#1}.} {#2} 
\end{leftbar} }

\renewenvironment{leftbar}{%
	\def\FrameCommand{\vrule width 1pt \relax\hspace {5pt}}
	\MakeFramed {\advance \hsize -\width \FrameRestore }
}{
	\endMakeFramed
}
\setlength{\parindent}{0in}
\setlength{\parskip}{0.2in}
\pagestyle{myheadings}

\renewcommand{\maketitle}{{\centering \LARGE \title{} \\
		\medskip\Large \author{} \\
		\smallskip \date{} \\
		%\emph{\large Assistance from \assistance}\\
	}
	\thispagestyle{empty}
	\vspace{0.5in}}

\renewcommand{\author}{Ryan Coopergard, David DeMark, Andy Hardt, Eric Stucky}
\renewcommand{\title}{Math 8272 Homework}
\renewcommand{\date}{2 May 2018}
\markright{\rm \title{} \hfill \author{} }


% --------------------------------------------------------------------


% quick blackboard bold
\newcommand{\n}{\mathbb{N}}
\newcommand{\z}{\mathbb{Z}}
\newcommand{\q}{\mathbb{Q}}
\newcommand{\p}{\mathbb{P}}
\renewcommand{\r}{\mathbb{R}}
\renewcommand{\c}{\mathbb{C}}
\renewcommand{\k}{\mathbb{K}}
\renewcommand{\t}{\mathbb{T}}

%Quick lowercase mathfrak
\newcommand{\afr}{\mathfrak{a}}
\newcommand{\bfr}{\mathfrak{b}}
\newcommand{\cfr}{\mathfrak{c}}
\newcommand{\dfr}{\mathfrak{d}}
\newcommand{\efr}{\mathfrak{e}}
\newcommand{\ffr}{\mathfrak{f}}
\newcommand{\gfr}{\mathfrak{g}}
\newcommand{\hfr}{\mathfrak{h}}
\newcommand{\ifr}{\mathfrak{i}}
\newcommand{\jfr}{\mathfrak{j}}
\newcommand{\kfr}{\mathfrak{k}}
\newcommand{\lfr}{\mathfrak{l}}
\newcommand{\mfr}{\mathfrak{m}}
\newcommand{\nfr}{\mathfrak{n}}
\newcommand{\ofr}{\mathfrak{o}}
\newcommand{\pfr}{\mathfrak{p}}
\newcommand{\qfr}{\mathfrak{q}}
\newcommand{\rfr}{\mathfrak{r}}
\newcommand{\sfr}{\mathfrak{s}}
\newcommand{\tfr}{\mathfrak{t}}
\newcommand{\ufr}{\mathfrak{u}}
\newcommand{\vfr}{\mathfrak{v}}
\newcommand{\wfr}{\mathfrak{w}}
\newcommand{\xfr}{\mathfrak{x}}
\newcommand{\yfr}{\mathfrak{y}}
\newcommand{\zfr}{\mathfrak{z}}

%Quick uppercase mathcal
\def\Acal{{\mathcal A}}
\def\Bcal{{\mathcal B}}
\def\Ccal{{\mathcal C}}
\def\Dcal{{\mathcal D}}
\def\Ecal{{\mathcal E}}
\def\Fcal{{\mathcal F}}
\def\Gcal{{\mathcal G}}
\def\Hcal{{\mathcal H}}
\def\Ical{{\mathcal I}}
\def\Jcal{{\mathcal J}}
\def\Lcal{{\mathcal L}}
\def\Mcal{{\mathcal M}}
\def\Ncal{{\mathcal N}}
\def\Ocal{{\mathcal O}}
\def\Pcal{{\mathcal P}}
\def\Qcal{{\mathcal Q}}
\def\Rcal{{\mathcal R}}
\def\Scal{{\mathcal S}}
\def\Tcal{{\mathcal T}}
\def\Xcal{{\mathcal X}}
\def\Zcal{{\mathcal Z}}

% quick greek letters
\renewcommand{\a}{\alpha}
\renewcommand{\b}{\beta}
\newcommand{\g}{\gamma}
\renewcommand{\d}{\delta}
\newcommand{\e}{\varepsilon}
\renewcommand{\l}{\lambda}

% quick linear algebra
\DeclareMathOperator{\im}{im}
\DeclareMathOperator{\rank}{rank}
\DeclareMathOperator{\tr}{tr}
\DeclareMathOperator{\vecspan}{span}
\newcommand{\<}{\left\langle}
\renewcommand{\>}{\right\rangle}

% quick algebra
\newcommand{\is}{\cong}
\newcommand{\iso}{\cong}
\newcommand{\filt}{\mathcal F}

\DeclareMathOperator{\id}{id}
\DeclareMathOperator{\Aut}{Aut}
\DeclareMathOperator{\End}{End}
\DeclareMathOperator{\rad}{rad}
\DeclareMathOperator{\initial}{in}

\DeclareMathOperator{\Hom}{Hom}
\renewcommand{\hom}[1]{\Hom_{{#1}}}
\DeclareMathOperator{\Ext}{Ext}
\newcommand{\ext}[1]{\Ext^{{#1}}}
\DeclareMathOperator{\Tor}{Tor}
\newcommand{\tor}[1]{\Tor^{{#1}}}
\DeclareMathOperator{\gr}{gr}
\newcommand{\Gr}[1]{\gr_{{#1}}}

% quick geometry
\DeclareMathOperator{\Spec}{Spec}
\renewcommand{\prime}{\mathfrak p}
\newcommand{\maxl}{\mathfrak m}
\newcommand{\affine}{\mathbb{A}}

% formatting conveniences
\newcommand{\ds}{\displaystyle}
\newcommand{\ts}{\textstyle}

\newcommand{\dsum}[4]{ \ds\sum_{#1 = #2}^{#3} {#4} }
\newcommand{\dset}[4]{ \ds\left\{ #4 \right\}_{#1 = #2}^{#3} }
\newcommand{\dseq}[4]{ \ds\left( #4 \right)_{#1 = #2}^{#3} }
\newcommand{\dcup}[4]{ \ds\bigcup_{#1 = #2}^{#3} #4 }
\newcommand{\dcap}[4]{ \ds\bigcap_{#1 = #2}^{#3} #4 }
\newcommand{\dint}[4]{\ds \int_{#2}^{#3}\!{#4}~\! \text{d}{#1}~\!}

\newcommand{\isum}[3]{ \ds\sum_{#1 \in #2} {#3} }
\newcommand{\iset}[3]{ \ds\left\{ #3 \right\}_{#1 \in #2}}
\newcommand{\iseq}[3]{ \ds\left( #3 \right)_{#1 \in #2}}
\newcommand{\icup}[3]{ \ds\bigcup_{#1 \in #2} #3 }
\newcommand{\icap}[3]{ \ds\bigcap_{#1 \in #2} #3 }
\renewcommand{\iint}[3]{\ds \int_{#2}\!{#3}~\! \text{d}{#1}~\!}

\newcommand{\pmat}[4]{ \ds\begin{bmatrix} {#1}&{#2}\\{#3}&{#4} \end{bmatrix} }
\newcommand{\smat}[9]{ \ds\begin{bmatrix} {#1}&{#2}&{#3} \\ {#4}&{#5}&{#6} \\ {#7}&{#8}&{#9} \end{bmatrix} }
\newcommand{\pvec}[2]{ \ds\begin{bmatrix} {#1}\\{#2} \end{bmatrix} }
\newcommand{\svec}[3]{ \ds\begin{bmatrix} {#1}\\{#2}\\{#3} \end{bmatrix} }
\newcommand{\phvec}[2]{ \ds\begin{bmatrix} {#1} & {#2} \end{bmatrix} }
\newcommand{\shvec}[3]{ \ds\begin{bmatrix} {#1} & {#2} & {#3} \end{bmatrix} }

% general laziness
\DeclareMathOperator{\supp}{supp}
\newcommand{\minus}{\smallsetminus}
\newcommand{\homeo}{\simeq}
\renewcommand{\ss}{\subseteq}
\newcommand{\pipe}[2]{\left.{#1}\right|_{#2}}

\DeclareMathOperator{\mor}{Mor}

\DeclareMathOperator{\coker}{coker}
\DeclareMathOperator{\ass}{Ass}
\DeclareMathOperator{\pd}{pd}

\newcommand{\col}[3]{{#1}:_{#2}{#3}}
\newcommand{\del}{\partial}
\DeclareMathOperator{\ann}{Ann}
\newcommand{\colim}{\varinjlim}
\newcommand{\limi}{\varprojlim}
\DeclareMathOperator{\spec}{Spec}
\DeclareMathOperator{\proj}{Proj}
\DeclareMathOperator{\mspec}{Max-Spec}

\DeclareMathOperator{\ins}{in}
\newcommand{\idl}[1]{\left\langle{#1}\right\rangle }

%%%Subproof environment
\newenvironment{subproof}[1][\proofname]{%
	\renewcommand{\qedsymbol}{$\blacksquare$}%
	\begin{proof}[#1]%
	}{%
	\end{proof}%
}


%%Problem Numbering(?) (I'm not married to any system in particular, but we should have an easy way to distinguish Eisenbud numbering and homework seheet numbering)
\usepackage{calc}
\newcommand{\prob}[1]{\setcounter{section}{#1-1}\section{}}
\newcommand{\prt}[1]{\setcounter{subsection}{#1-1}\subsection{}}
\renewcommand\thesubsection{\alph{subsection}}
\usepackage[sl,bf,compact]{titlesec}
\titlelabel{\thetitle.)\quad}
%%%

%%%%%Probelm-internal theorem labeling (by using the \prob command, you set the section numbering to the problem number--thus if you want to include a sub-proposition in your proof, the first one is labeled "Proposition [problem #].A")

\newtheorem{theorem}{Theorem}[section]
\newtheorem*{thm*}{Theorem}
\newtheorem{cor}[theorem]{Corollary}
\newtheorem*{cor*}{Corollary}
\newtheorem{lemma}[theorem]{Lemma}
\newtheorem*{lemma*}{Lemma}
\newtheorem{prop}[theorem]{Proposition}
\newtheorem*{prop*}{Proposition}
\newtheorem{claim}[theorem]{Claim}
\newtheorem*{claim*}{Claim}
\theoremstyle{definition}
\newtheorem{definition}[theorem]{Definition}
\newtheorem{notation}[theorem]{Notation}
\newtheorem*{notation*}{Notation}
\renewcommand{\thetheorem}{\arabic{section}.\Alph{theorem}}

\usepackage[textsize=small]{todonotes}
%\usepackage[disable]{todonotes}

%%%
%%%%%%%%%%COMMENT MACROS%%%%%%%%%%
%feel free to change your color-coding of course


%David
\newcommand{\ddt}[1]{\todo{#1}~}
\newcommand{\ddi}[1]{\todo[inline]{#1}~}
\newcommand{\ddf}[1]{\todo[fancyline]{#1}~}

%Andy
\newcommand{\aht}[1]{\todo[color=yellow]{#1}~}
\newcommand{\ahi}[1]{\todo[inline,color=yellow]{#1}~}
\newcommand{\ahf}[1]{\todo[fancyline,color=yellow]{#1}~}

%Eric
\newcommand{\est}[1]{\todo[color=lightgray]{#1}~}
\newcommand{\esi}[1]{\todo[inline,color=lightgray]{#1}~}
\newcommand{\esf}[1]{\todo[fancyline,color=lightgray]{#1}~}

%Ryan
\newcommand{\rct}[1]{\todo[color=cyan]{#1}~}
\newcommand{\rci}[1]{\todo[inline,color=cyan]{#1}~}
\newcommand{\rcf}[1]{\todo[fancyline,color=cyan]{#1}~}


% --------------------------------------------------------------------


% --------------------------------------------------------------------


\begin{document}
	\maketitle{}
	
	\ddi{So with Eric's preamble, \textbackslash prime is re-def'd to be \textbackslash mathfrak \{p\}, which causes the apostrophe to be interpreted as "to the mathfrak p." is there a way around this? or are we okay with letting mathfrak p be referred to as \textbackslash pfr? }
	\prob{4}
	\problem{Proposition (Eisenbud Ex. 9.2)}{There exists an infinite-dimensional Noetherian ring.}
	\begin{proof}
		We let\vspace*{-1em}\textellipsis\begin{itemize}
			
			\item $k$ be any field, \item $R=k[x_1,x_2,\ldots]$, \item $d:\n_0\to \n$ a strictly increasing function with first difference function $\delta: \n\to \n$ defined by $s(m)=d(m)-d(m-1)$ such that $d(0)=1$ and $s$ is strictly increasing as well, \item $P_m=\idl{x_{d(m-1)},x_{d(m-1)+1},\ldots,x_{d(m)}}$ for $m\geq 1$,\item $U$ be the multiplicative system $\left(\bigcup_{m=1}^\infty P_m\right)^c$,\item and $S$ be the ring $U^{-1}R$.
		\end{itemize}
		We shall now show that $\dim S=\infty$, but $S$ is Noetherian. We break this argument down into a series of claims.
		\begin{prop}[Eisenbud, Ex. 3.14]\label{4:claim:maxs}
			The maximal ideals of $S$ are precisely the ideals $P_m$.
		\end{prop}
		\begin{subproof}[Proof of Proposition \ref{4:claim:maxs}]
			We let $I$ be a proper ideal of $S$ (noting that necessarily, $I\subset \bigcup_{m=1}^\infty P_m$) and $0\neq f\in I$ an arbitrary element. We let $\Acal_f:=\{P_{i_1},\hdots,P_{i_n}\}:=\{P_i\;:\;P_i \text{ contains a monomial of } f\}$. We let $g\neq f$ be another arbitrary element of $I$ and suppose for the sake of contradiction that $g$ has some monomial term $g'$ such that $g'\notin \bigcup_{j=1}^nP_{i_j}$. Then, $f+g$ has a nonzero coefficient for $g'$. As each $P_m$ is a monomial ideal and hence contains all monomials of each of its elements, we now have that for any $P_{i_k}\in \Acal_f$, $f+g\notin P_{i_k}$. However, by an identical argument, for any $P_j\ni g'$, $f+g\notin P_j$, since $f$ necessarily has monomial terms not in $P_j$. Returning to the monomial ideal argument, we have now shown that $f+g\notin \bigcup_{m=1}^\infty P_m$, thus inducing a contradiction. Thus, for any ideal $I\subset S$, we have that $I\subset \bigcup_{k=1}^N P_{j_N}$ for some finite $\{j_1,\hdots,J_N\}$. Prime avoidance then implies $I\subset P_M$ for some $M\in \n$. As it is the case that $P_m\centernot\subseteq P_{m'}$ for $m\neq m'$, this completes our proof.
		\end{subproof}
		Next, as suggested by the text, we prove Eisenbud's lemma 9.4.
		\begin{lemma}[Eisenbud, Lemma 9.4]\label{4:lem:9.4}
			Let $Q$ be a ring with the properties (i) for any maximal $\mfr\subset Q$, $Q_\mfr$ is Noetherian and (ii) each element $s\in Q$ is contained in finitely many maximal ideals. Then, $Q$ is Noetherian.
		\end{lemma}
		\begin{subproof}[Proof of Lemma \ref{4:lem:9.4}]
			We suppose for the sake of contradiction that there exists an infinite chain of ideals $0=I_0\subsetneq I_1\subsetneq I_2\subsetneq\ldots$ in $Q$. We then define the function $N:\mspec(Q)\to \n_0$ by $\mfr\mapsto \min\{n\;:\; I_n\centernot \subseteq\mfr\}$. As each $Q_\mfr$ is Noetherian, we must have that $N(\mfr)$ exists and is finite. We also define the choice function $C:\n_0\to \mspec(Q)$ which assigns to each $n\in \n_0$ some $\mfr\in \mspec(Q)$ such that $I_n\subset \mfr$. As each ideal of a ring must be contained in a maximal ideal by Zorn's lemma, there exists some well-defined such $C$. We observe that $C(N(\mfr))\neq \mfr$ for any $\mfr\in \mspec(Q)$ as $I_{N(\mfr)}\not \subset \mfr$ by construction. We also observe that $n\leq N(C(n))$, as $I_m\subset C(n)$ for any $m\leq n$ but $I_{N(C(n))}\not\subseteq C(n)$ similarly by construction. We now iteratively define a sequence of distinct maximal ideals $\{\mfr_1,\mfr_2,\hdots\}$ by letting $\mfr_1:=C(1)$ and for $i>1$, $\mfr_i:=C(N(\mfr_{i-1}))$. As $N\circ C$ has been shown to be a strictly increasing function, we have by well-ordering that for any $n$, there exists a $J$ such that $I_n\subset \mfr_j$ for all $j>J$. However, then $I_n\subset \bigcap_{j=J}^\infty \mfr_j$, contradicting our assumptions on $Q$.  
		\end{subproof}
		\begin{cor}
			$S$ is Noetherian
		\end{cor}
	\end{proof}
	
	
	Problem 5: (I'm not quite sure how our formatting works).
	Krull dimension satisfies the first half of axiom D1, and also the axiom D2. In other words, \[\dim R = \sup_{P\subset R \text{prime}} \dim R_P\] and if $I$ is a nilpotent ideal, then $\dim R = \dim R/I$.
	
	Proof: If $P$ is a prime ideal of $R$, let $P_0\subset\ldots\subset P_n$ be a chain of primes in $R_P$. If $\phi$ is the natural map from $R\to R_P$, then Proposition 2.2 of Eisenbud tells us that $P_i = \phi^{-1}(P_i)R_P$. The ideal $\phi^{-1}(P_i)\subset R$ is prime because if the complement of $\phi^{-1}(P_i)$ weren't multiplicatively closed, then the map $\phi$ would tell us that the complement of $P_i$ was also not multiplicatively closed. In addition, if $P_i\subsetneq P_j$, then $\phi^{-1}(P_i)R_P \subsetneq \phi^{-1}(P_j)R_P$, so $\phi^{-1}(P_i)\subsetneq \phi^{-1}(P_j)$. Therefore, any chain of primes in $R_P$ lifts to an equal length chain in $R$.
	
	On the other hand, let $P_1,P_2,\ldots$ be a sequence of primes in $R$ such that $\dim P_i\to\dim R$. This is possible because for a finite chain with minimal prime $Q$, $\dim Q$ is the length of that chain, and for an infinite chain, by taking smaller and smaller primes in the chain, we get such a sequence. If $P_i\subset Q_{i1}\subset Q_{i2}\ldots$ is a chain in $R$ starting with $P_i$ (i.e. a chain corresponding to one in $R/P_i$), then it will be a chain of the same length in $R_{P_i}$. Thus we have that $\dim R_{P_i}\ge \dim P_i$, so $\sup_{P\subset R \text{prime}} \dim R_P\ge \dim R$.
	
	Now, if $I$ is nilpotent, we have $\dim R\ge \dim R/I$, the fourth isomorphism theorem gives us a correspondence between prime ideals of $R/I$ and prime ideals of $R$ containing $I$. Now $I$ is contained in the nilradical of $R$, so it is contained in every prime of $R$, so chains of primes of $R$ are in one-to-one correspondence with chains of primes of $R/I$, so $\dim R = \dim R/i$.\qed
	
	
	OLD PROBLEMS FOR FORMATTING CHECKS
	
	
	%====%
	\problem{Theorem 2.7}{
		Show that the universal property of localization is unique up to unique isomorphism; that is, if another $R\to S$ has the same property....
	}
	
	
	\problem{Theorem 2.4}{
		Let $R=k[x]$. Describe as explicitly as possible:
		\begin{enumerate}
			\item $\hom{\z}(\z_n,\z_m)$ and $\hom{R}(R/(x^n), R/(x^m))$,
			\item $\z_n\otimes \z_m$ and $R/(x^n)\otimes R/(x^m)$,
			\item $R\otimes_k R$ (describe this as an algebra).
		\end{enumerate}
	}
	
	
	%====%
	\problem{Theorem 3.17}{
		Show that if $k=\z_2$ then the ideal $(x,y)\ss k[x,y]/(x,y)^2$ is the union of three properly smaller ideals.
		
		Let $k$ be any field, and $I_1=(x), I_2=(y)$ and $J=(x^2,y)$ ideals in the ring $k[x,y]/(xy,y^2)$. Show that the homogeneous elements of $J$ are contained in $I_1\cup I_2$, but that $J\not\ss I_1,I_2$. (Note that one of the $I$ is prime.)
	}
	
	
	\problem{Theorem 3.6--8}{
		Which monomial ideals are prime? Irreducible? Radical? Primary?
		
		Find an algorithm for computing the radical of a monomial ideal.
		
		Find an algorithm for computing an irreducible decomposition, and thus a primary decomposition, of a monomial ideal.
	}
	
	
	%====%
	\problem{Theorem 4.7}{
		Show the Jacobsen radical of $R$ is $\{r: 1+rs \text{ is a unit for every } s\in R\}$.
	}
	
	
	\problem{Theorem 4.11}{
		\begin{enumerate}
			\item Use Nakayama's lemma to show that if $R$ is local and $M$ is finitely generated projective, then $M$ is free. If $R$ is a positively graded ring, with $R_0$ a field, and $M$ is a finitely generated graded projective, then $M$ is a graded free module.
			\item Use Prop 2.10 (contains the snippet $\hom{S}\otimes{R}\hom{R}(M,N)\is \hom{S}\otimes_R M, S\otimes_R N)$) to show that a finitely presente module $M$ is projective iff $M$ is locally free in the sense that localization $M_P$ is free over $R_P$ for every maximal ideal of $R$.
		\end{enumerate}
	}
	
	
	\problem{Theorem 4.24}{
		Let $R$ be either of the domains $\c[x,y]/(y^2-x^3)$ or $\c[x,y](y^2-x^2(x+1))$ and let $t=y/x$ an element of the quotient field. Show that in each case, $R[t]=\c[t]$.
	}
	
	
	\problem{Theorem 4.26}{
		Suppose that the additive group of $R$ is a finitely generated abelian group. If $P$ is a maximal ideal of $R$, show that $R/P$ is a finite field. Show that every prime ideal of $R$ that is not maximal is a minimal prime ideal.
	}\rcf{Ryan Comment}
	
	
	%====%
	\problem{Theorem 5.1}{
		Let $R$ be a ring and $M$ be an $R$-module. Suppose that $\cdots \ss M_1\ss M_0=M$ is a filtration by submodules. Although the map $M\to \gr M$ sending $f$ to $\initial(f)$ is not a homomorphism of abelian groups, show that  $\initial(f)+\initial(g)$ is either $\initial(f+g)$ or $0$.
		
		Moreover, suppose that $M=R$ and the filtration is multiplicative. Show that $\initial(f)\initial(g)$ is either $\initial(fg)$ or $0$.
	}\est{Eric Comment}
	
	
	\problem{Theorem 5.8}{
		\begin{enumerate}
			\item Let $R=k[x,y]/(x^2-y^3)$, and let $I=(x,y)$. Show that $R$ is a domain, but $\initial(x)^2=0$ in $\gr_I(R)$.
			\item Let $R=k[t^4,t^5,t^{11}]\ss k[t]$, and let $I=(t^4,t^5,t^{11})$. Show that $\initial(I)\initial(t^{11})=0$
		\end{enumerate}
	}\ahi{Andy Comment}
	
	
	
	%====%
	\problem{Theorem 6.1}{
		Let $R$ be a ring and $M$ an $R$-module. Show that $M$ is flat iff $\Tor_1(M,N)=0$ for all $R$-modules $N$ iff $\Tor(N,M)=0$ for all $R$-modules $N$ iff $\Tor_i=0$ for all $R$-modules $N$ and all $i>0$.
	}\ddi{David Comment}
	
	
	
	%====%
	\problem{Theorem 7.11}{
		Let $R$ be N\"otherian, and $\maxl = (a_1,\dots, a_n)$ be an ideal. Show that
		$$\hat{R}_\maxl \is R[[x_1,\dots, x_n]]/(x_1-a_1,\dots, x_n-a_n).$$
	}
	
	
	
	%====%
	\problem{Theorem A3.6}{
		Let $R$ be N\"otherian and $M$ be any finitely generate $R$-module.
		\begin{enumerate}
			\item Let $P$ be prime. Show that if $M\to E(R/P)$ is any map [into the injective envelope], then $\ker\a$ is a $P$-primary submodule of $M$.
			\item Show that $E(M)$ is a finite direct sum of indecomposable projectives. Let $M\to E(M)=\oplus E(R/P_i)$, and show that if $P$ is a prime ideal and $M(P)$ is the kernel of the composite map $M\to E(M)\to \oplus_{P_i=P} E(R/P_i)$, then $M(P)$ is $P$-primary. Show that $0=\cap M(P)$ is a primary decomposition of zero, and that the set of $P$ that occurs among the $P_i$ above is precisely the set $\text{Ass}(M)$.
		\end{enumerate}
	}
	
	
	\problem{Theorem A3.13}{
		Show that if $0\to N_f\to F\to M\to 0$ and $0\to N_G\to G\to M\to 0$ are exact with $F$ and $G$ projective, then $N_F\oplus G \is N_G\oplus F$ and both are $\ker(F\oplus G\to M)$.
	}
	
	
	\problem{Theorem A3.18}{
		Let $(R,\maxl)$ be a local ring. We say that a free reslution $(F_i, \varphi_i)$ is minimal if each $\varphi_i$ has an image contained in $\maxl F_{i-1}$.  If $F$ as above is a minial free resolution of $M$ and $\rank F_i=b_i$, then show that $\Tor_i(R/\maxl)\is (R/\maxl)^{b_i}$. [The $b_i$ are called Betti numbers of $M$, in loose analogy with the situation in topology where $F$ is a chain complex.]
	}
	
	
	\problem{Theorem A3.23}{
		If $x$ is not a zero-divisor in a ring $R$, compute $\Ext^i(R/x,M)$. In particular, compute $\Ext^i(\z_n,\z_m)$ for any integers $n,m$.
	}
	
	
	\problem{Theorem A3.24}{
		Show that a finitely generated abelian group $A$ is free iff $\Ext_\z(a,\z)=0$. It was conjectured that this would hold for all groups, but Shelah proved in 1974 that this depends on your set theory.
	}
	
	
	
	
\end{document}

% If any of you wants to put in a decent header, be my guest.
% This is just the one that I stole from my undergrad :P 
% The top section is all somewhat essential, the bottom is just
%   a random dump of commands I've used once.   -eric

% --------------------------------------------------------------------

\documentclass[Letter,12pt]{article}
\usepackage{amsmath,amssymb,amsthm}
\usepackage{graphicx,framed,enumerate}
\usepackage[margin=1in]{geometry}
\usepackage{fancyhdr}
\usepackage{microtype}
\usepackage{centernot}
\usepackage{dirtytalk}
\usepackage{enumitem}
\frenchspacing

\newcommand{\problem}[2]{
	\vspace{0.3in} 
	\begin{leftbar} 
		\noindent \textbf{{#1}.} {#2} 
\end{leftbar} }

\renewenvironment{leftbar}{%
	\def\FrameCommand{\vrule width 1pt \relax\hspace {5pt}}
	\MakeFramed {\advance \hsize -\width \FrameRestore }
}{
	\endMakeFramed
}
\setlength{\parindent}{0in}
\setlength{\parskip}{0.2in}
\pagestyle{myheadings}

\renewcommand{\maketitle}{{\centering \LARGE \title{} \\
		\medskip\Large \author{} \\
		\smallskip \date{} \\
		%\emph{\large Assistance from \assistance}\\
	}
	\thispagestyle{empty}
	\vspace{0.5in}}

\renewcommand{\author}{Ryan Coopergard, David DeMark, Andy Hardt, Eric Stucky}
\renewcommand{\title}{Math 8272 Homework}
\renewcommand{\date}{2 May 2018}
\markright{\rm \title{} \hfill \author{} }


% --------------------------------------------------------------------


% quick blackboard bold
\newcommand{\n}{\mathbb{N}}
\newcommand{\z}{\mathbb{Z}}
\newcommand{\q}{\mathbb{Q}}
\newcommand{\p}{\mathbb{P}}
\renewcommand{\r}{\mathbb{R}}
\renewcommand{\c}{\mathbb{C}}
\renewcommand{\k}{\mathbb{K}}
\renewcommand{\t}{\mathbb{T}}

%Quick lowercase mathfrak
\newcommand{\afr}{\mathfrak{a}}
\newcommand{\bfr}{\mathfrak{b}}
\newcommand{\cfr}{\mathfrak{c}}
\newcommand{\dfr}{\mathfrak{d}}
\newcommand{\efr}{\mathfrak{e}}
\newcommand{\ffr}{\mathfrak{f}}
\newcommand{\gfr}{\mathfrak{g}}
\newcommand{\hfr}{\mathfrak{h}}
\newcommand{\ifr}{\mathfrak{i}}
\newcommand{\jfr}{\mathfrak{j}}
\newcommand{\kfr}{\mathfrak{k}}
\newcommand{\lfr}{\mathfrak{l}}
\newcommand{\mfr}{\mathfrak{m}}
\newcommand{\nfr}{\mathfrak{n}}
\newcommand{\ofr}{\mathfrak{o}}
\newcommand{\pfr}{\mathfrak{p}}
\newcommand{\qfr}{\mathfrak{q}}
\newcommand{\rfr}{\mathfrak{r}}
\newcommand{\sfr}{\mathfrak{s}}
\newcommand{\tfr}{\mathfrak{t}}
\newcommand{\ufr}{\mathfrak{u}}
\newcommand{\vfr}{\mathfrak{v}}
\newcommand{\wfr}{\mathfrak{w}}
\newcommand{\xfr}{\mathfrak{x}}
\newcommand{\yfr}{\mathfrak{y}}
\newcommand{\zfr}{\mathfrak{z}}

%Quick uppercase mathcal
\def\Acal{{\mathcal A}}
\def\Bcal{{\mathcal B}}
\def\Ccal{{\mathcal C}}
\def\Dcal{{\mathcal D}}
\def\Ecal{{\mathcal E}}
\def\Fcal{{\mathcal F}}
\def\Gcal{{\mathcal G}}
\def\Hcal{{\mathcal H}}
\def\Ical{{\mathcal I}}
\def\Jcal{{\mathcal J}}
\def\Lcal{{\mathcal L}}
\def\Mcal{{\mathcal M}}
\def\Ncal{{\mathcal N}}
\def\Ocal{{\mathcal O}}
\def\Pcal{{\mathcal P}}
\def\Qcal{{\mathcal Q}}
\def\Rcal{{\mathcal R}}
\def\Scal{{\mathcal S}}
\def\Tcal{{\mathcal T}}
\def\Xcal{{\mathcal X}}
\def\Zcal{{\mathcal Z}}

% quick greek letters
\renewcommand{\a}{\alpha}
\renewcommand{\b}{\beta}
\newcommand{\g}{\gamma}
\renewcommand{\d}{\delta}
\newcommand{\e}{\varepsilon}
\renewcommand{\l}{\lambda}

% quick linear algebra
\DeclareMathOperator{\im}{im}
\DeclareMathOperator{\rank}{rank}
\DeclareMathOperator{\tr}{tr}
\DeclareMathOperator{\vecspan}{span}
\newcommand{\<}{\left\langle}
\renewcommand{\>}{\right\rangle}

% quick algebra
\newcommand{\is}{\cong}
\newcommand{\iso}{\cong}
\newcommand{\filt}{\mathcal F}

\DeclareMathOperator{\id}{id}
\DeclareMathOperator{\Aut}{Aut}
\DeclareMathOperator{\End}{End}
\DeclareMathOperator{\rad}{rad}
\DeclareMathOperator{\initial}{in}

\DeclareMathOperator{\Hom}{Hom}
\renewcommand{\hom}[1]{\Hom_{{#1}}}
\DeclareMathOperator{\Ext}{Ext}
\newcommand{\ext}[1]{\Ext^{{#1}}}
\DeclareMathOperator{\Tor}{Tor}
\newcommand{\tor}[1]{\Tor^{{#1}}}
\DeclareMathOperator{\gr}{gr}
\newcommand{\Gr}[1]{\gr_{{#1}}}

% quick geometry
\DeclareMathOperator{\Spec}{Spec}
\renewcommand{\prime}{\mathfrak p}
\newcommand{\maxl}{\mathfrak m}
\newcommand{\affine}{\mathbb{A}}

% formatting conveniences
\newcommand{\ds}{\displaystyle}
\newcommand{\ts}{\textstyle}

\newcommand{\dsum}[4]{ \ds\sum_{#1 = #2}^{#3} {#4} }
\newcommand{\dset}[4]{ \ds\left\{ #4 \right\}_{#1 = #2}^{#3} }
\newcommand{\dseq}[4]{ \ds\left( #4 \right)_{#1 = #2}^{#3} }
\newcommand{\dcup}[4]{ \ds\bigcup_{#1 = #2}^{#3} #4 }
\newcommand{\dcap}[4]{ \ds\bigcap_{#1 = #2}^{#3} #4 }
\newcommand{\dint}[4]{\ds \int_{#2}^{#3}\!{#4}~\! \text{d}{#1}~\!}

\newcommand{\isum}[3]{ \ds\sum_{#1 \in #2} {#3} }
\newcommand{\iset}[3]{ \ds\left\{ #3 \right\}_{#1 \in #2}}
\newcommand{\iseq}[3]{ \ds\left( #3 \right)_{#1 \in #2}}
\newcommand{\icup}[3]{ \ds\bigcup_{#1 \in #2} #3 }
\newcommand{\icap}[3]{ \ds\bigcap_{#1 \in #2} #3 }
\renewcommand{\iint}[3]{\ds \int_{#2}\!{#3}~\! \text{d}{#1}~\!}

\newcommand{\pmat}[4]{ \ds\begin{bmatrix} {#1}&{#2}\\{#3}&{#4} \end{bmatrix} }
\newcommand{\smat}[9]{ \ds\begin{bmatrix} {#1}&{#2}&{#3} \\ {#4}&{#5}&{#6} \\ {#7}&{#8}&{#9} \end{bmatrix} }
\newcommand{\pvec}[2]{ \ds\begin{bmatrix} {#1}\\{#2} \end{bmatrix} }
\newcommand{\svec}[3]{ \ds\begin{bmatrix} {#1}\\{#2}\\{#3} \end{bmatrix} }
\newcommand{\phvec}[2]{ \ds\begin{bmatrix} {#1} & {#2} \end{bmatrix} }
\newcommand{\shvec}[3]{ \ds\begin{bmatrix} {#1} & {#2} & {#3} \end{bmatrix} }

% general laziness
\DeclareMathOperator{\supp}{supp}
\newcommand{\minus}{\smallsetminus}
\newcommand{\homeo}{\simeq}
\renewcommand{\ss}{\subseteq}
\newcommand{\pipe}[2]{\left.{#1}\right|_{#2}}

\DeclareMathOperator{\mor}{Mor}

\DeclareMathOperator{\coker}{coker}
\DeclareMathOperator{\ass}{Ass}
\DeclareMathOperator{\pd}{pd}

\newcommand{\col}[3]{{#1}:_{#2}{#3}}
\newcommand{\del}{\partial}
\DeclareMathOperator{\ann}{Ann}
\newcommand{\colim}{\varinjlim}
\newcommand{\limi}{\varprojlim}
\DeclareMathOperator{\spec}{Spec}
\DeclareMathOperator{\proj}{Proj}
\DeclareMathOperator{\mspec}{Max-Spec}

\DeclareMathOperator{\ins}{in}
\newcommand{\idl}[1]{\left\langle{#1}\right\rangle }

%%%Subproof environment
\newenvironment{subproof}[1][\proofname]{%
	\renewcommand{\qedsymbol}{$\blacksquare$}%
	\begin{proof}[#1]%
	}{%
	\end{proof}%
}


%%Problem Numbering(?) (I'm not married to any system in particular, but we should have an easy way to distinguish Eisenbud numbering and homework seheet numbering)
\usepackage{calc}
\newcommand{\prob}[1]{\setcounter{section}{#1-1}\section{}}
\newcommand{\prt}[1]{\setcounter{subsection}{#1-1}\subsection{}}
\renewcommand\thesubsection{\alph{subsection}}
\usepackage[sl,bf,compact]{titlesec}
\titlelabel{\thetitle.)\quad}
%%%

%%%%%Probelm-internal theorem labeling (by using the \prob command, you set the section numbering to the problem number--thus if you want to include a sub-proposition in your proof, the first one is labeled "Proposition [problem #].A")

\newtheorem{theorem}{Theorem}[section]
\newtheorem*{thm*}{Theorem}
\newtheorem{cor}[theorem]{Corollary}
\newtheorem*{cor*}{Corollary}
\newtheorem{lemma}[theorem]{Lemma}
\newtheorem*{lemma*}{Lemma}
\newtheorem{prop}[theorem]{Proposition}
\newtheorem*{prop*}{Proposition}
\newtheorem{claim}[theorem]{Claim}
\newtheorem*{claim*}{Claim}
\theoremstyle{definition}
\newtheorem{definition}[theorem]{Definition}
\newtheorem{notation}[theorem]{Notation}
\newtheorem*{notation*}{Notation}
\renewcommand{\thetheorem}{\arabic{section}.\Alph{theorem}}

\usepackage[textsize=small]{todonotes}
%\usepackage[disable]{todonotes}

%%%
%%%%%%%%%%COMMENT MACROS%%%%%%%%%%
%feel free to change your color-coding of course


%David
\newcommand{\ddt}[1]{\todo{#1}~}
\newcommand{\ddi}[1]{\todo[inline]{#1}~}
\newcommand{\ddf}[1]{\todo[fancyline]{#1}~}

%Andy
\newcommand{\aht}[1]{\todo[color=yellow]{#1}~}
\newcommand{\ahi}[1]{\todo[inline,color=yellow]{#1}~}
\newcommand{\ahf}[1]{\todo[fancyline,color=yellow]{#1}~}

%Eric
\newcommand{\est}[1]{\todo[color=lightgray]{#1}~}
\newcommand{\esi}[1]{\todo[inline,color=lightgray]{#1}~}
\newcommand{\esf}[1]{\todo[fancyline,color=lightgray]{#1}~}

%Ryan
\newcommand{\rct}[1]{\todo[color=cyan]{#1}~}
\newcommand{\rci}[1]{\todo[inline,color=cyan]{#1}~}
\newcommand{\rcf}[1]{\todo[fancyline,color=cyan]{#1}~}


% --------------------------------------------------------------------


% --------------------------------------------------------------------


\begin{document}
	\maketitle{}
	
	\ddi{So with Eric's preamble, \textbackslash prime is re-def'd to be \textbackslash mathfrak \{p\}, which causes the apostrophe to be interpreted as "to the mathfrak p." is there a way around this? or are we okay with letting mathfrak p be referred to as \textbackslash pfr? }
	\prob{4}
	\problem{Proposition (Eisenbud Ex. 9.2)}{There exists an infinite-dimensional Noetherian ring.}
	\begin{proof}
		We let\vspace*{-1em}\textellipsis\begin{itemize}
			
			\item $k$ be any field, \item $R=k[x_1,x_2,\ldots]$, \item $d:\n_0\to \n$ a strictly increasing function with first difference function $\delta: \n\to \n$ defined by $s(m)=d(m)-d(m-1)$ such that $d(0)=1$ and $s$ is strictly increasing as well, \item $P_m=\idl{x_{d(m-1)},x_{d(m-1)+1},\ldots,x_{d(m)}}$ for $m\geq 1$,\item $U$ be the multiplicative system $\left(\bigcup_{m=1}^\infty P_m\right)^c$,\item and $S$ be the ring $U^{-1}R$.
		\end{itemize}
		We shall now show that $\dim S=\infty$, but $S$ is Noetherian. We break this argument down into a series of claims.
		\begin{prop}[Eisenbud, Ex. 3.14]\label{4:claim:maxs}
			The maximal ideals of $S$ are precisely the ideals $P_m$.
		\end{prop}
		\begin{subproof}[Proof of Proposition \ref{4:claim:maxs}]
			We let $I$ be a proper ideal of $S$ (noting that necessarily, $I\subset \bigcup_{m=1}^\infty P_m$) and $0\neq f\in I$ an arbitrary element. We let $\Acal_f:=\{P_{i_1},\hdots,P_{i_n}\}:=\{P_i\;:\;P_i \text{ contains a monomial of } f\}$. We let $g\neq f$ be another arbitrary element of $I$ and suppose for the sake of contradiction that $g$ has some monomial term $g'$ such that $g'\notin \bigcup_{j=1}^nP_{i_j}$. Then, $f+g$ has a nonzero coefficient for $g'$. As each $P_m$ is a monomial ideal and hence contains all monomials of each of its elements, we now have that for any $P_{i_k}\in \Acal_f$, $f+g\notin P_{i_k}$. However, by an identical argument, for any $P_j\ni g'$, $f+g\notin P_j$, since $f$ necessarily has monomial terms not in $P_j$. Returning to the monomial ideal argument, we have now shown that $f+g\notin \bigcup_{m=1}^\infty P_m$, thus inducing a contradiction. Thus, for any ideal $I\subset S$, we have that $I\subset \bigcup_{k=1}^N P_{j_N}$ for some finite $\{j_1,\hdots,J_N\}$. Prime avoidance then implies $I\subset P_M$ for some $M\in \n$. As it is the case that $P_m\centernot\subseteq P_{m'}$ for $m\neq m'$, this completes our proof.
		\end{subproof}
		Next, as suggested by the text, we prove Eisenbud's lemma 9.4.
		\begin{lemma}[Eisenbud, Lemma 9.4]\label{4:lem:9.4}
			Let $Q$ be a ring with the properties (i) for any maximal $\mfr\subset Q$, $Q_\mfr$ is Noetherian and (ii) each element $s\in Q$ is contained in finitely many maximal ideals. Then, $Q$ is Noetherian.
		\end{lemma}
		\begin{subproof}[Proof of Lemma \ref{4:lem:9.4}]
			We suppose for the sake of contradiction that there exists an infinite chain of ideals $0=I_0\subsetneq I_1\subsetneq I_2\subsetneq\ldots$ in $Q$. We then define the function $N:\mspec(Q)\to \n_0$ by $\mfr\mapsto \min\{n\;:\; I_n\centernot \subseteq\mfr\}$. As each $Q_\mfr$ is Noetherian, we must have that $N(\mfr)$ exists and is finite. We also define the choice function $C:\n_0\to \mspec(Q)$ which assigns to each $n\in \n_0$ some $\mfr\in \mspec(Q)$ such that $I_n\subset \mfr$. As each ideal of a ring must be contained in a maximal ideal by Zorn's lemma, there exists some well-defined such $C$. We observe that $C(N(\mfr))\neq \mfr$ for any $\mfr\in \mspec(Q)$ as $I_{N(\mfr)}\not \subset \mfr$ by construction. We also observe that $n\leq N(C(n))$, as $I_m\subset C(n)$ for any $m\leq n$ but $I_{N(C(n))}\not\subseteq C(n)$ similarly by construction. We now iteratively define a sequence of distinct maximal ideals $\{\mfr_1,\mfr_2,\hdots\}$ by letting $\mfr_1:=C(1)$ and for $i>1$, $\mfr_i:=C(N(\mfr_{i-1}))$. As $N\circ C$ has been shown to be a strictly increasing function, we have by well-ordering that for any $n$, there exists a $J$ such that $I_n\subset \mfr_j$ for all $j>J$. However, then $I_n\subset \bigcap_{j=J}^\infty \mfr_j$, contradicting our assumptions on $Q$.  
		\end{subproof}
		\begin{cor}
			$S$ is Noetherian
		\end{cor}
	\end{proof}
	
	
	Problem 5: (I'm not quite sure how our formatting works).
	Krull dimension satisfies the first half of axiom D1, and also the axiom D2. In other words, \[\dim R = \sup_{P\subset R \text{prime}} \dim R_P\] and if $I$ is a nilpotent ideal, then $\dim R = \dim R/I$.
	
	Proof: If $P$ is a prime ideal of $R$, let $P_0\subset\ldots\subset P_n$ be a chain of primes in $R_P$. If $\phi$ is the natural map from $R\to R_P$, then Proposition 2.2 of Eisenbud tells us that $P_i = \phi^{-1}(P_i)R_P$. The ideal $\phi^{-1}(P_i)\subset R$ is prime because if the complement of $\phi^{-1}(P_i)$ weren't multiplicatively closed, then the map $\phi$ would tell us that the complement of $P_i$ was also not multiplicatively closed. In addition, if $P_i\subsetneq P_j$, then $\phi^{-1}(P_i)R_P \subsetneq \phi^{-1}(P_j)R_P$, so $\phi^{-1}(P_i)\subsetneq \phi^{-1}(P_j)$. Therefore, any chain of primes in $R_P$ lifts to an equal length chain in $R$.
	
	On the other hand, let $P_1,P_2,\ldots$ be a sequence of primes in $R$ such that $\dim P_i\to\dim R$. This is possible because for a finite chain with minimal prime $Q$, $\dim Q$ is the length of that chain, and for an infinite chain, by taking smaller and smaller primes in the chain, we get such a sequence. If $P_i\subset Q_{i1}\subset Q_{i2}\ldots$ is a chain in $R$ starting with $P_i$ (i.e. a chain corresponding to one in $R/P_i$), then it will be a chain of the same length in $R_{P_i}$. Thus we have that $\dim R_{P_i}\ge \dim P_i$, so $\sup_{P\subset R \text{prime}} \dim R_P\ge \dim R$.
	
	Now, if $I$ is nilpotent, we have $\dim R\ge \dim R/I$, the fourth isomorphism theorem gives us a correspondence between prime ideals of $R/I$ and prime ideals of $R$ containing $I$. Now $I$ is contained in the nilradical of $R$, so it is contained in every prime of $R$, so chains of primes of $R$ are in one-to-one correspondence with chains of primes of $R/I$, so $\dim R = \dim R/i$.\qed
	
	
	OLD PROBLEMS FOR FORMATTING CHECKS
	
	
	%====%
	\problem{Theorem 2.7}{
		Show that the universal property of localization is unique up to unique isomorphism; that is, if another $R\to S$ has the same property....
	}
	
	
	\problem{Theorem 2.4}{
		Let $R=k[x]$. Describe as explicitly as possible:
		\begin{enumerate}
			\item $\hom{\z}(\z_n,\z_m)$ and $\hom{R}(R/(x^n), R/(x^m))$,
			\item $\z_n\otimes \z_m$ and $R/(x^n)\otimes R/(x^m)$,
			\item $R\otimes_k R$ (describe this as an algebra).
		\end{enumerate}
	}
	
	
	%====%
	\problem{Theorem 3.17}{
		Show that if $k=\z_2$ then the ideal $(x,y)\ss k[x,y]/(x,y)^2$ is the union of three properly smaller ideals.
		
		Let $k$ be any field, and $I_1=(x), I_2=(y)$ and $J=(x^2,y)$ ideals in the ring $k[x,y]/(xy,y^2)$. Show that the homogeneous elements of $J$ are contained in $I_1\cup I_2$, but that $J\not\ss I_1,I_2$. (Note that one of the $I$ is prime.)
	}
	
	
	\problem{Theorem 3.6--8}{
		Which monomial ideals are prime? Irreducible? Radical? Primary?
		
		Find an algorithm for computing the radical of a monomial ideal.
		
		Find an algorithm for computing an irreducible decomposition, and thus a primary decomposition, of a monomial ideal.
	}
	
	
	%====%
	\problem{Theorem 4.7}{
		Show the Jacobsen radical of $R$ is $\{r: 1+rs \text{ is a unit for every } s\in R\}$.
	}
	
	
	\problem{Theorem 4.11}{
		\begin{enumerate}
			\item Use Nakayama's lemma to show that if $R$ is local and $M$ is finitely generated projective, then $M$ is free. If $R$ is a positively graded ring, with $R_0$ a field, and $M$ is a finitely generated graded projective, then $M$ is a graded free module.
			\item Use Prop 2.10 (contains the snippet $\hom{S}\otimes{R}\hom{R}(M,N)\is \hom{S}\otimes_R M, S\otimes_R N)$) to show that a finitely presente module $M$ is projective iff $M$ is locally free in the sense that localization $M_P$ is free over $R_P$ for every maximal ideal of $R$.
		\end{enumerate}
	}
	
	
	\problem{Theorem 4.24}{
		Let $R$ be either of the domains $\c[x,y]/(y^2-x^3)$ or $\c[x,y](y^2-x^2(x+1))$ and let $t=y/x$ an element of the quotient field. Show that in each case, $R[t]=\c[t]$.
	}
	
	
	\problem{Theorem 4.26}{
		Suppose that the additive group of $R$ is a finitely generated abelian group. If $P$ is a maximal ideal of $R$, show that $R/P$ is a finite field. Show that every prime ideal of $R$ that is not maximal is a minimal prime ideal.
	}\rcf{Ryan Comment}
	
	
	%====%
	\problem{Theorem 5.1}{
		Let $R$ be a ring and $M$ be an $R$-module. Suppose that $\cdots \ss M_1\ss M_0=M$ is a filtration by submodules. Although the map $M\to \gr M$ sending $f$ to $\initial(f)$ is not a homomorphism of abelian groups, show that  $\initial(f)+\initial(g)$ is either $\initial(f+g)$ or $0$.
		
		Moreover, suppose that $M=R$ and the filtration is multiplicative. Show that $\initial(f)\initial(g)$ is either $\initial(fg)$ or $0$.
	}\est{Eric Comment}
	
	
	\problem{Theorem 5.8}{
		\begin{enumerate}
			\item Let $R=k[x,y]/(x^2-y^3)$, and let $I=(x,y)$. Show that $R$ is a domain, but $\initial(x)^2=0$ in $\gr_I(R)$.
			\item Let $R=k[t^4,t^5,t^{11}]\ss k[t]$, and let $I=(t^4,t^5,t^{11})$. Show that $\initial(I)\initial(t^{11})=0$
		\end{enumerate}
	}\ahi{Andy Comment}
	
	
	
	%====%
	\problem{Theorem 6.1}{
		Let $R$ be a ring and $M$ an $R$-module. Show that $M$ is flat iff $\Tor_1(M,N)=0$ for all $R$-modules $N$ iff $\Tor(N,M)=0$ for all $R$-modules $N$ iff $\Tor_i=0$ for all $R$-modules $N$ and all $i>0$.
	}\ddi{David Comment}
	
	
	
	%====%
	\problem{Theorem 7.11}{
		Let $R$ be N\"otherian, and $\maxl = (a_1,\dots, a_n)$ be an ideal. Show that
		$$\hat{R}_\maxl \is R[[x_1,\dots, x_n]]/(x_1-a_1,\dots, x_n-a_n).$$
	}
	
	
	
	%====%
	\problem{Theorem A3.6}{
		Let $R$ be N\"otherian and $M$ be any finitely generate $R$-module.
		\begin{enumerate}
			\item Let $P$ be prime. Show that if $M\to E(R/P)$ is any map [into the injective envelope], then $\ker\a$ is a $P$-primary submodule of $M$.
			\item Show that $E(M)$ is a finite direct sum of indecomposable projectives. Let $M\to E(M)=\oplus E(R/P_i)$, and show that if $P$ is a prime ideal and $M(P)$ is the kernel of the composite map $M\to E(M)\to \oplus_{P_i=P} E(R/P_i)$, then $M(P)$ is $P$-primary. Show that $0=\cap M(P)$ is a primary decomposition of zero, and that the set of $P$ that occurs among the $P_i$ above is precisely the set $\text{Ass}(M)$.
		\end{enumerate}
	}
	
	
	\problem{Theorem A3.13}{
		Show that if $0\to N_f\to F\to M\to 0$ and $0\to N_G\to G\to M\to 0$ are exact with $F$ and $G$ projective, then $N_F\oplus G \is N_G\oplus F$ and both are $\ker(F\oplus G\to M)$.
	}
	
	
	\problem{Theorem A3.18}{
		Let $(R,\maxl)$ be a local ring. We say that a free reslution $(F_i, \varphi_i)$ is minimal if each $\varphi_i$ has an image contained in $\maxl F_{i-1}$.  If $F$ as above is a minial free resolution of $M$ and $\rank F_i=b_i$, then show that $\Tor_i(R/\maxl)\is (R/\maxl)^{b_i}$. [The $b_i$ are called Betti numbers of $M$, in loose analogy with the situation in topology where $F$ is a chain complex.]
	}
	
	
	\problem{Theorem A3.23}{
		If $x$ is not a zero-divisor in a ring $R$, compute $\Ext^i(R/x,M)$. In particular, compute $\Ext^i(\z_n,\z_m)$ for any integers $n,m$.
	}
	
	
	\problem{Theorem A3.24}{
		Show that a finitely generated abelian group $A$ is free iff $\Ext_\z(a,\z)=0$. It was conjectured that this would hold for all groups, but Shelah proved in 1974 that this depends on your set theory.
	}
	
	
	
	
\end{document}

% If any of you wants to put in a decent header, be my guest.
% This is just the one that I stole from my undergrad :P 
% The top section is all somewhat essential, the bottom is just
%   a random dump of commands I've used once.   -eric

% --------------------------------------------------------------------

\documentclass[Letter,12pt]{article}
\usepackage{amsmath,amssymb,amsthm}
\usepackage{graphicx,framed,enumerate}
\usepackage[margin=1in]{geometry}
\usepackage{fancyhdr}
\usepackage{microtype}
\usepackage{centernot}
\usepackage{dirtytalk}
\usepackage{enumitem}
\frenchspacing

\newcommand{\problem}[2]{
	\vspace{0.3in} 
	\begin{leftbar} 
		\noindent \textbf{{#1}.} {#2} 
\end{leftbar} }

\renewenvironment{leftbar}{%
	\def\FrameCommand{\vrule width 1pt \relax\hspace {5pt}}
	\MakeFramed {\advance \hsize -\width \FrameRestore }
}{
	\endMakeFramed
}
\setlength{\parindent}{0in}
\setlength{\parskip}{0.2in}
\pagestyle{myheadings}

\renewcommand{\maketitle}{{\centering \LARGE \title{} \\
		\medskip\Large \author{} \\
		\smallskip \date{} \\
		%\emph{\large Assistance from \assistance}\\
	}
	\thispagestyle{empty}
	\vspace{0.5in}}

\renewcommand{\author}{Ryan Coopergard, David DeMark, Andy Hardt, Eric Stucky}
\renewcommand{\title}{Math 8272 Homework}
\renewcommand{\date}{2 May 2018}
\markright{\rm \title{} \hfill \author{} }


% --------------------------------------------------------------------


% quick blackboard bold
\newcommand{\n}{\mathbb{N}}
\newcommand{\z}{\mathbb{Z}}
\newcommand{\q}{\mathbb{Q}}
\newcommand{\p}{\mathbb{P}}
\renewcommand{\r}{\mathbb{R}}
\renewcommand{\c}{\mathbb{C}}
\renewcommand{\k}{\mathbb{K}}
\renewcommand{\t}{\mathbb{T}}

%Quick lowercase mathfrak
\newcommand{\afr}{\mathfrak{a}}
\newcommand{\bfr}{\mathfrak{b}}
\newcommand{\cfr}{\mathfrak{c}}
\newcommand{\dfr}{\mathfrak{d}}
\newcommand{\efr}{\mathfrak{e}}
\newcommand{\ffr}{\mathfrak{f}}
\newcommand{\gfr}{\mathfrak{g}}
\newcommand{\hfr}{\mathfrak{h}}
\newcommand{\ifr}{\mathfrak{i}}
\newcommand{\jfr}{\mathfrak{j}}
\newcommand{\kfr}{\mathfrak{k}}
\newcommand{\lfr}{\mathfrak{l}}
\newcommand{\mfr}{\mathfrak{m}}
\newcommand{\nfr}{\mathfrak{n}}
\newcommand{\ofr}{\mathfrak{o}}
\newcommand{\pfr}{\mathfrak{p}}
\newcommand{\qfr}{\mathfrak{q}}
\newcommand{\rfr}{\mathfrak{r}}
\newcommand{\sfr}{\mathfrak{s}}
\newcommand{\tfr}{\mathfrak{t}}
\newcommand{\ufr}{\mathfrak{u}}
\newcommand{\vfr}{\mathfrak{v}}
\newcommand{\wfr}{\mathfrak{w}}
\newcommand{\xfr}{\mathfrak{x}}
\newcommand{\yfr}{\mathfrak{y}}
\newcommand{\zfr}{\mathfrak{z}}

%Quick uppercase mathcal
\def\Acal{{\mathcal A}}
\def\Bcal{{\mathcal B}}
\def\Ccal{{\mathcal C}}
\def\Dcal{{\mathcal D}}
\def\Ecal{{\mathcal E}}
\def\Fcal{{\mathcal F}}
\def\Gcal{{\mathcal G}}
\def\Hcal{{\mathcal H}}
\def\Ical{{\mathcal I}}
\def\Jcal{{\mathcal J}}
\def\Lcal{{\mathcal L}}
\def\Mcal{{\mathcal M}}
\def\Ncal{{\mathcal N}}
\def\Ocal{{\mathcal O}}
\def\Pcal{{\mathcal P}}
\def\Qcal{{\mathcal Q}}
\def\Rcal{{\mathcal R}}
\def\Scal{{\mathcal S}}
\def\Tcal{{\mathcal T}}
\def\Xcal{{\mathcal X}}
\def\Zcal{{\mathcal Z}}

% quick greek letters
\renewcommand{\a}{\alpha}
\renewcommand{\b}{\beta}
\newcommand{\g}{\gamma}
\renewcommand{\d}{\delta}
\newcommand{\e}{\varepsilon}
\renewcommand{\l}{\lambda}

% quick linear algebra
\DeclareMathOperator{\im}{im}
\DeclareMathOperator{\rank}{rank}
\DeclareMathOperator{\tr}{tr}
\DeclareMathOperator{\vecspan}{span}
\newcommand{\<}{\left\langle}
\renewcommand{\>}{\right\rangle}

% quick algebra
\newcommand{\is}{\cong}
\newcommand{\iso}{\cong}
\newcommand{\filt}{\mathcal F}

\DeclareMathOperator{\id}{id}
\DeclareMathOperator{\Aut}{Aut}
\DeclareMathOperator{\End}{End}
\DeclareMathOperator{\rad}{rad}
\DeclareMathOperator{\initial}{in}

\DeclareMathOperator{\Hom}{Hom}
\renewcommand{\hom}[1]{\Hom_{{#1}}}
\DeclareMathOperator{\Ext}{Ext}
\newcommand{\ext}[1]{\Ext^{{#1}}}
\DeclareMathOperator{\Tor}{Tor}
\newcommand{\tor}[1]{\Tor^{{#1}}}
\DeclareMathOperator{\gr}{gr}
\newcommand{\Gr}[1]{\gr_{{#1}}}

% quick geometry
\DeclareMathOperator{\Spec}{Spec}
\renewcommand{\prime}{\mathfrak p}
\newcommand{\maxl}{\mathfrak m}
\newcommand{\affine}{\mathbb{A}}

% formatting conveniences
\newcommand{\ds}{\displaystyle}
\newcommand{\ts}{\textstyle}

\newcommand{\dsum}[4]{ \ds\sum_{#1 = #2}^{#3} {#4} }
\newcommand{\dset}[4]{ \ds\left\{ #4 \right\}_{#1 = #2}^{#3} }
\newcommand{\dseq}[4]{ \ds\left( #4 \right)_{#1 = #2}^{#3} }
\newcommand{\dcup}[4]{ \ds\bigcup_{#1 = #2}^{#3} #4 }
\newcommand{\dcap}[4]{ \ds\bigcap_{#1 = #2}^{#3} #4 }
\newcommand{\dint}[4]{\ds \int_{#2}^{#3}\!{#4}~\! \text{d}{#1}~\!}

\newcommand{\isum}[3]{ \ds\sum_{#1 \in #2} {#3} }
\newcommand{\iset}[3]{ \ds\left\{ #3 \right\}_{#1 \in #2}}
\newcommand{\iseq}[3]{ \ds\left( #3 \right)_{#1 \in #2}}
\newcommand{\icup}[3]{ \ds\bigcup_{#1 \in #2} #3 }
\newcommand{\icap}[3]{ \ds\bigcap_{#1 \in #2} #3 }
\renewcommand{\iint}[3]{\ds \int_{#2}\!{#3}~\! \text{d}{#1}~\!}

\newcommand{\pmat}[4]{ \ds\begin{bmatrix} {#1}&{#2}\\{#3}&{#4} \end{bmatrix} }
\newcommand{\smat}[9]{ \ds\begin{bmatrix} {#1}&{#2}&{#3} \\ {#4}&{#5}&{#6} \\ {#7}&{#8}&{#9} \end{bmatrix} }
\newcommand{\pvec}[2]{ \ds\begin{bmatrix} {#1}\\{#2} \end{bmatrix} }
\newcommand{\svec}[3]{ \ds\begin{bmatrix} {#1}\\{#2}\\{#3} \end{bmatrix} }
\newcommand{\phvec}[2]{ \ds\begin{bmatrix} {#1} & {#2} \end{bmatrix} }
\newcommand{\shvec}[3]{ \ds\begin{bmatrix} {#1} & {#2} & {#3} \end{bmatrix} }

% general laziness
\DeclareMathOperator{\supp}{supp}
\newcommand{\minus}{\smallsetminus}
\newcommand{\homeo}{\simeq}
\renewcommand{\ss}{\subseteq}
\newcommand{\pipe}[2]{\left.{#1}\right|_{#2}}

\DeclareMathOperator{\mor}{Mor}

\DeclareMathOperator{\coker}{coker}
\DeclareMathOperator{\ass}{Ass}
\DeclareMathOperator{\pd}{pd}

\newcommand{\col}[3]{{#1}:_{#2}{#3}}
\newcommand{\del}{\partial}
\DeclareMathOperator{\ann}{Ann}
\newcommand{\colim}{\varinjlim}
\newcommand{\limi}{\varprojlim}
\DeclareMathOperator{\spec}{Spec}
\DeclareMathOperator{\proj}{Proj}
\DeclareMathOperator{\mspec}{Max-Spec}

\DeclareMathOperator{\ins}{in}
\newcommand{\idl}[1]{\left\langle{#1}\right\rangle }

%%%Subproof environment
\newenvironment{subproof}[1][\proofname]{%
	\renewcommand{\qedsymbol}{$\blacksquare$}%
	\begin{proof}[#1]%
	}{%
	\end{proof}%
}


%%Problem Numbering(?) (I'm not married to any system in particular, but we should have an easy way to distinguish Eisenbud numbering and homework seheet numbering)
\usepackage{calc}
\newcommand{\prob}[1]{\setcounter{section}{#1-1}\section{}}
\newcommand{\prt}[1]{\setcounter{subsection}{#1-1}\subsection{}}
\renewcommand\thesubsection{\alph{subsection}}
\usepackage[sl,bf,compact]{titlesec}
\titlelabel{\thetitle.)\quad}
%%%

%%%%%Probelm-internal theorem labeling (by using the \prob command, you set the section numbering to the problem number--thus if you want to include a sub-proposition in your proof, the first one is labeled "Proposition [problem #].A")

\newtheorem{theorem}{Theorem}[section]
\newtheorem*{thm*}{Theorem}
\newtheorem{cor}[theorem]{Corollary}
\newtheorem*{cor*}{Corollary}
\newtheorem{lemma}[theorem]{Lemma}
\newtheorem*{lemma*}{Lemma}
\newtheorem{prop}[theorem]{Proposition}
\newtheorem*{prop*}{Proposition}
\newtheorem{claim}[theorem]{Claim}
\newtheorem*{claim*}{Claim}
\theoremstyle{definition}
\newtheorem{definition}[theorem]{Definition}
\newtheorem{notation}[theorem]{Notation}
\newtheorem*{notation*}{Notation}
\renewcommand{\thetheorem}{\arabic{section}.\Alph{theorem}}

\usepackage[textsize=small]{todonotes}
%\usepackage[disable]{todonotes}

%%%
%%%%%%%%%%COMMENT MACROS%%%%%%%%%%
%feel free to change your color-coding of course


%David
\newcommand{\ddt}[1]{\todo{#1}~}
\newcommand{\ddi}[1]{\todo[inline]{#1}~}
\newcommand{\ddf}[1]{\todo[fancyline]{#1}~}

%Andy
\newcommand{\aht}[1]{\todo[color=yellow]{#1}~}
\newcommand{\ahi}[1]{\todo[inline,color=yellow]{#1}~}
\newcommand{\ahf}[1]{\todo[fancyline,color=yellow]{#1}~}

%Eric
\newcommand{\est}[1]{\todo[color=lightgray]{#1}~}
\newcommand{\esi}[1]{\todo[inline,color=lightgray]{#1}~}
\newcommand{\esf}[1]{\todo[fancyline,color=lightgray]{#1}~}

%Ryan
\newcommand{\rct}[1]{\todo[color=cyan]{#1}~}
\newcommand{\rci}[1]{\todo[inline,color=cyan]{#1}~}
\newcommand{\rcf}[1]{\todo[fancyline,color=cyan]{#1}~}


% --------------------------------------------------------------------


% --------------------------------------------------------------------


\begin{document}
	\maketitle{}
	
	\ddi{So with Eric's preamble, \textbackslash prime is re-def'd to be \textbackslash mathfrak \{p\}, which causes the apostrophe to be interpreted as "to the mathfrak p." is there a way around this? or are we okay with letting mathfrak p be referred to as \textbackslash pfr? }
	\prob{4}
	\problem{Proposition (Eisenbud Ex. 9.2)}{There exists an infinite-dimensional Noetherian ring.}
	\begin{proof}
		We let\vspace*{-1em}\textellipsis\begin{itemize}
			
			\item $k$ be any field, \item $R=k[x_1,x_2,\ldots]$, \item $d:\n_0\to \n$ a strictly increasing function with first difference function $\delta: \n\to \n$ defined by $s(m)=d(m)-d(m-1)$ such that $d(0)=1$ and $s$ is strictly increasing as well, \item $P_m=\idl{x_{d(m-1)},x_{d(m-1)+1},\ldots,x_{d(m)}}$ for $m\geq 1$,\item $U$ be the multiplicative system $\left(\bigcup_{m=1}^\infty P_m\right)^c$,\item and $S$ be the ring $U^{-1}R$.
		\end{itemize}
		We shall now show that $\dim S=\infty$, but $S$ is Noetherian. We break this argument down into a series of claims.
		\begin{prop}[Eisenbud, Ex. 3.14]\label{4:claim:maxs}
			The maximal ideals of $S$ are precisely the ideals $P_m$.
		\end{prop}
		\begin{subproof}[Proof of Proposition \ref{4:claim:maxs}]
			We let $I$ be a proper ideal of $S$ (noting that necessarily, $I\subset \bigcup_{m=1}^\infty P_m$) and $0\neq f\in I$ an arbitrary element. We let $\Acal_f:=\{P_{i_1},\hdots,P_{i_n}\}:=\{P_i\;:\;P_i \text{ contains a monomial of } f\}$. We let $g\neq f$ be another arbitrary element of $I$ and suppose for the sake of contradiction that $g$ has some monomial term $g'$ such that $g'\notin \bigcup_{j=1}^nP_{i_j}$. Then, $f+g$ has a nonzero coefficient for $g'$. As each $P_m$ is a monomial ideal and hence contains all monomials of each of its elements, we now have that for any $P_{i_k}\in \Acal_f$, $f+g\notin P_{i_k}$. However, by an identical argument, for any $P_j\ni g'$, $f+g\notin P_j$, since $f$ necessarily has monomial terms not in $P_j$. Returning to the monomial ideal argument, we have now shown that $f+g\notin \bigcup_{m=1}^\infty P_m$, thus inducing a contradiction. Thus, for any ideal $I\subset S$, we have that $I\subset \bigcup_{k=1}^N P_{j_N}$ for some finite $\{j_1,\hdots,J_N\}$. Prime avoidance then implies $I\subset P_M$ for some $M\in \n$. As it is the case that $P_m\centernot\subseteq P_{m'}$ for $m\neq m'$, this completes our proof.
		\end{subproof}
		Next, as suggested by the text, we prove Eisenbud's lemma 9.4.
		\begin{lemma}[Eisenbud, Lemma 9.4]\label{4:lem:9.4}
			Let $Q$ be a ring with the properties (i) for any maximal $\mfr\subset Q$, $Q_\mfr$ is Noetherian and (ii) each element $s\in Q$ is contained in finitely many maximal ideals. Then, $Q$ is Noetherian.
		\end{lemma}
		\begin{subproof}[Proof of Lemma \ref{4:lem:9.4}]
			We suppose for the sake of contradiction that there exists an infinite chain of ideals $0=I_0\subsetneq I_1\subsetneq I_2\subsetneq\ldots$ in $Q$. We then define the function $N:\mspec(Q)\to \n_0$ by $\mfr\mapsto \min\{n\;:\; I_n\centernot \subseteq\mfr\}$. As each $Q_\mfr$ is Noetherian, we must have that $N(\mfr)$ exists and is finite. We also define the choice function $C:\n_0\to \mspec(Q)$ which assigns to each $n\in \n_0$ some $\mfr\in \mspec(Q)$ such that $I_n\subset \mfr$. As each ideal of a ring must be contained in a maximal ideal by Zorn's lemma, there exists some well-defined such $C$. We observe that $C(N(\mfr))\neq \mfr$ for any $\mfr\in \mspec(Q)$ as $I_{N(\mfr)}\not \subset \mfr$ by construction. We also observe that $n\leq N(C(n))$, as $I_m\subset C(n)$ for any $m\leq n$ but $I_{N(C(n))}\not\subseteq C(n)$ similarly by construction. We now iteratively define a sequence of distinct maximal ideals $\{\mfr_1,\mfr_2,\hdots\}$ by letting $\mfr_1:=C(1)$ and for $i>1$, $\mfr_i:=C(N(\mfr_{i-1}))$. As $N\circ C$ has been shown to be a strictly increasing function, we have by well-ordering that for any $n$, there exists a $J$ such that $I_n\subset \mfr_j$ for all $j>J$. However, then $I_n\subset \bigcap_{j=J}^\infty \mfr_j$, contradicting our assumptions on $Q$. 
		\end{subproof}
		\begin{cor}
	\label{4:cor:Noe}		$S$ is Noetherian.
		\end{cor}
	\begin{subproof}[Proof of Corollary \ref{4:cor:Noe}]
		What is left to show is that $Q$ fulfills the hypotheses of Lemma \ref{4:lem:9.4}.
	\end{subproof}
	\end{proof}
	
	
	Problem 5: (I'm not quite sure how our formatting works).
	Krull dimension satisfies the first half of axiom D1, and also the axiom D2. In other words, \[\dim R = \sup_{P\subset R \text{prime}} \dim R_P\] and if $I$ is a nilpotent ideal, then $\dim R = \dim R/I$.
	
	Proof: If $P$ is a prime ideal of $R$, let $P_0\subset\ldots\subset P_n$ be a chain of primes in $R_P$. If $\phi$ is the natural map from $R\to R_P$, then Proposition 2.2 of Eisenbud tells us that $P_i = \phi^{-1}(P_i)R_P$. The ideal $\phi^{-1}(P_i)\subset R$ is prime because if the complement of $\phi^{-1}(P_i)$ weren't multiplicatively closed, then the map $\phi$ would tell us that the complement of $P_i$ was also not multiplicatively closed. In addition, if $P_i\subsetneq P_j$, then $\phi^{-1}(P_i)R_P \subsetneq \phi^{-1}(P_j)R_P$, so $\phi^{-1}(P_i)\subsetneq \phi^{-1}(P_j)$. Therefore, any chain of primes in $R_P$ lifts to an equal length chain in $R$.
	
	On the other hand, let $P_1,P_2,\ldots$ be a sequence of primes in $R$ such that $\dim P_i\to\dim R$. This is possible because for a finite chain with minimal prime $Q$, $\dim Q$ is the length of that chain, and for an infinite chain, by taking smaller and smaller primes in the chain, we get such a sequence. If $P_i\subset Q_{i1}\subset Q_{i2}\ldots$ is a chain in $R$ starting with $P_i$ (i.e. a chain corresponding to one in $R/P_i$), then it will be a chain of the same length in $R_{P_i}$. Thus we have that $\dim R_{P_i}\ge \dim P_i$, so $\sup_{P\subset R \text{prime}} \dim R_P\ge \dim R$.
	
	Now, if $I$ is nilpotent, we have $\dim R\ge \dim R/I$, the fourth isomorphism theorem gives us a correspondence between prime ideals of $R/I$ and prime ideals of $R$ containing $I$. Now $I$ is contained in the nilradical of $R$, so it is contained in every prime of $R$, so chains of primes of $R$ are in one-to-one correspondence with chains of primes of $R/I$, so $\dim R = \dim R/i$.\qed
	
	
	OLD PROBLEMS FOR FORMATTING CHECKS
	
	
	%====%
	\problem{Theorem 2.7}{
		Show that the universal property of localization is unique up to unique isomorphism; that is, if another $R\to S$ has the same property....
	}
	
	
	\problem{Theorem 2.4}{
		Let $R=k[x]$. Describe as explicitly as possible:
		\begin{enumerate}
			\item $\hom{\z}(\z_n,\z_m)$ and $\hom{R}(R/(x^n), R/(x^m))$,
			\item $\z_n\otimes \z_m$ and $R/(x^n)\otimes R/(x^m)$,
			\item $R\otimes_k R$ (describe this as an algebra).
		\end{enumerate}
	}
	
	
	%====%
	\problem{Theorem 3.17}{
		Show that if $k=\z_2$ then the ideal $(x,y)\ss k[x,y]/(x,y)^2$ is the union of three properly smaller ideals.
		
		Let $k$ be any field, and $I_1=(x), I_2=(y)$ and $J=(x^2,y)$ ideals in the ring $k[x,y]/(xy,y^2)$. Show that the homogeneous elements of $J$ are contained in $I_1\cup I_2$, but that $J\not\ss I_1,I_2$. (Note that one of the $I$ is prime.)
	}
	
	
	\problem{Theorem 3.6--8}{
		Which monomial ideals are prime? Irreducible? Radical? Primary?
		
		Find an algorithm for computing the radical of a monomial ideal.
		
		Find an algorithm for computing an irreducible decomposition, and thus a primary decomposition, of a monomial ideal.
	}
	
	
	%====%
	\problem{Theorem 4.7}{
		Show the Jacobsen radical of $R$ is $\{r: 1+rs \text{ is a unit for every } s\in R\}$.
	}
	
	
	\problem{Theorem 4.11}{
		\begin{enumerate}
			\item Use Nakayama's lemma to show that if $R$ is local and $M$ is finitely generated projective, then $M$ is free. If $R$ is a positively graded ring, with $R_0$ a field, and $M$ is a finitely generated graded projective, then $M$ is a graded free module.
			\item Use Prop 2.10 (contains the snippet $\hom{S}\otimes{R}\hom{R}(M,N)\is \hom{S}\otimes_R M, S\otimes_R N)$) to show that a finitely presente module $M$ is projective iff $M$ is locally free in the sense that localization $M_P$ is free over $R_P$ for every maximal ideal of $R$.
		\end{enumerate}
	}
	
	
	\problem{Theorem 4.24}{
		Let $R$ be either of the domains $\c[x,y]/(y^2-x^3)$ or $\c[x,y](y^2-x^2(x+1))$ and let $t=y/x$ an element of the quotient field. Show that in each case, $R[t]=\c[t]$.
	}
	
	
	\problem{Theorem 4.26}{
		Suppose that the additive group of $R$ is a finitely generated abelian group. If $P$ is a maximal ideal of $R$, show that $R/P$ is a finite field. Show that every prime ideal of $R$ that is not maximal is a minimal prime ideal.
	}\rcf{Ryan Comment}
	
	
	%====%
	\problem{Theorem 5.1}{
		Let $R$ be a ring and $M$ be an $R$-module. Suppose that $\cdots \ss M_1\ss M_0=M$ is a filtration by submodules. Although the map $M\to \gr M$ sending $f$ to $\initial(f)$ is not a homomorphism of abelian groups, show that  $\initial(f)+\initial(g)$ is either $\initial(f+g)$ or $0$.
		
		Moreover, suppose that $M=R$ and the filtration is multiplicative. Show that $\initial(f)\initial(g)$ is either $\initial(fg)$ or $0$.
	}\est{Eric Comment}
	
	
	\problem{Theorem 5.8}{
		\begin{enumerate}
			\item Let $R=k[x,y]/(x^2-y^3)$, and let $I=(x,y)$. Show that $R$ is a domain, but $\initial(x)^2=0$ in $\gr_I(R)$.
			\item Let $R=k[t^4,t^5,t^{11}]\ss k[t]$, and let $I=(t^4,t^5,t^{11})$. Show that $\initial(I)\initial(t^{11})=0$
		\end{enumerate}
	}\ahi{Andy Comment}
	
	
	
	%====%
	\problem{Theorem 6.1}{
		Let $R$ be a ring and $M$ an $R$-module. Show that $M$ is flat iff $\Tor_1(M,N)=0$ for all $R$-modules $N$ iff $\Tor(N,M)=0$ for all $R$-modules $N$ iff $\Tor_i=0$ for all $R$-modules $N$ and all $i>0$.
	}\ddi{David Comment}
	
	
	
	%====%
	\problem{Theorem 7.11}{
		Let $R$ be N\"otherian, and $\maxl = (a_1,\dots, a_n)$ be an ideal. Show that
		$$\hat{R}_\maxl \is R[[x_1,\dots, x_n]]/(x_1-a_1,\dots, x_n-a_n).$$
	}
	
	
	
	%====%
	\problem{Theorem A3.6}{
		Let $R$ be N\"otherian and $M$ be any finitely generate $R$-module.
		\begin{enumerate}
			\item Let $P$ be prime. Show that if $M\to E(R/P)$ is any map [into the injective envelope], then $\ker\a$ is a $P$-primary submodule of $M$.
			\item Show that $E(M)$ is a finite direct sum of indecomposable projectives. Let $M\to E(M)=\oplus E(R/P_i)$, and show that if $P$ is a prime ideal and $M(P)$ is the kernel of the composite map $M\to E(M)\to \oplus_{P_i=P} E(R/P_i)$, then $M(P)$ is $P$-primary. Show that $0=\cap M(P)$ is a primary decomposition of zero, and that the set of $P$ that occurs among the $P_i$ above is precisely the set $\text{Ass}(M)$.
		\end{enumerate}
	}
	
	
	\problem{Theorem A3.13}{
		Show that if $0\to N_f\to F\to M\to 0$ and $0\to N_G\to G\to M\to 0$ are exact with $F$ and $G$ projective, then $N_F\oplus G \is N_G\oplus F$ and both are $\ker(F\oplus G\to M)$.
	}
	
	
	\problem{Theorem A3.18}{
		Let $(R,\maxl)$ be a local ring. We say that a free reslution $(F_i, \varphi_i)$ is minimal if each $\varphi_i$ has an image contained in $\maxl F_{i-1}$.  If $F$ as above is a minial free resolution of $M$ and $\rank F_i=b_i$, then show that $\Tor_i(R/\maxl)\is (R/\maxl)^{b_i}$. [The $b_i$ are called Betti numbers of $M$, in loose analogy with the situation in topology where $F$ is a chain complex.]
	}
	
	
	\problem{Theorem A3.23}{
		If $x$ is not a zero-divisor in a ring $R$, compute $\Ext^i(R/x,M)$. In particular, compute $\Ext^i(\z_n,\z_m)$ for any integers $n,m$.
	}
	
	
	\problem{Theorem A3.24}{
		Show that a finitely generated abelian group $A$ is free iff $\Ext_\z(a,\z)=0$. It was conjectured that this would hold for all groups, but Shelah proved in 1974 that this depends on your set theory.
	}
	
	
	
	
\end{document}

